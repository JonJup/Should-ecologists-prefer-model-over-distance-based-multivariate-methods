% !TeX spellcheck = en_US
\section{Introduction}

	%------------------------------%
	%&			Erste Absatz:
	%&	Eternal Question + Statisticians point of view 
	%------------------------------%

	Which environmental gradients control species abundances and community composition, is one of the most essential questions in ecology \citep[e.g.][]{Clements1907}  and the current alteration of ecosystems at an unprecedented rate endows it with a new urgency \citep{pacifici2015assessing}. 
	%
	Given the complexity of simulating ecological systems under artificial conditions (e.g. in microcosms), monitoring the abundance or occurrence of taxa across sites with variable  environmental conditions has been one approach to tackle this question.
	%
	Related studies deliver a sites-by-species matrix $\mathbf{Y}$ containing multivariate species abundances, which is then statistically related to  a sites-by-predictor matrix $\mathbf{X}$, containing information on the environmental predictors.
	%
	From a statistical perspective, $\mathbf{Y}$ has many undesirable properties such as
	intercorrelations between variables, 
	e.g. through biotic interactions \citep{morales2015inferring},
	probability distributions other than the normal, 
	more species than sites \citep[\textit{high dimensionality}, especially in DNA barcoding studies, ][]{cristescu2014barcoding},  
	and many zeros, because most species are commonly absent from most sites (\textit{sparsity}) \citep{mcgill2007species}. 
	
	%------------------------------%
	%&		Zweite Absatz:
	%&	Distance Based + Whats wrong with it
	%------------------------------%
    
    While univariate data are routinely analyzed by model-based methods like ANOVA, generalized linear models and linear mixed models, multivariate data are most often analyzed with algorithm-based methods.
    %
    Most of the latter are variations on three basic approaches: i) correspondence analysis, ii) redundancy analysis, or iii) analysis of pairwise distance matrices. 
    %
    Algorithm-based methods have weak links to ecological theory and do not take the data's statistical properties into account. 
    %
    In distance-based algorithms, for example, the researcher implicitly assumes a mean-variance relationship of the data by choosing a certain distance metric. 
	% 
	For instance, Minkowski distances (e.g. Manhattan and Euclidean) assume a constant variance across all mean values \citep{TerBraak1988}
	%
	whereas species abundances often show a quadratic mean-variance relationship \citep{routledge1991taylor, yamamura1999transformation}.
	%
	Whether a distance metric is appropriate depends on the properties of the data and the aim of the study, as each metric extracts different information from the raw data. 
	%
	The choice is complicated by the vast amount of available metrics \citep[see][]{Legendre2012}.
	%
	Misspecification of the relationship by choosing an inappropriate distance metric can lead to erroneous conclusions as demonstrated by \citet{Warton2012}.
	%
	An alternative to algorithm-based analyses that accounts for mean-variance relationships and incorporates ecological assumptions is the model-based approach.\\
 
	%------------------------------%
	%&		Dritte Absatz:
	%&	Model Based Approach
	%------------------------------%

    % -- This is mirrored in the history of CCA and CQO         

	The model-based approach consists of explicitly specifying a statistical model of the process that generated the observed data \citep{Warton2015a}.
	%
	This includes properties such as marginal distributions, overdispersion, zero-inflation, mean-variance relationship, and correlation structure, all of which can be flexibly tailored to the data and the research question.
	%
	While this approach is ubiquitous in univariate analyses \citep[e.g.][]{bolker2008ecological, zuur2010protocol}, it has long been uncommon in multivariate ecological analyses, largely due to the absence of suitable models \citep{anderson2001new}.
	%
    However, advances in statistical theory and computation power have led to a surge of models for multivariate abundance data. 
	%
	Recent examples include Hierarchical Modeling of Species Communities \citep{Ovaskainen2017}, Generalized Joint 
	%
	Attribute Modelling \citep{Clark2017},  and multivariate Generalized Linear Models \citep[MvGLM,][]{Warton2012}.
	
	%------------------------------%
	%&		Vierte Absatz:
	%&			GLMmv
	%------------------------------%
	
	In MvGLM, a separate univariate GLM is fit to each taxon, with each model using the same predictors. 
	%
	Univariate GLMs are a flexible method and are strongly advocated for the analysis of count or occurrence data as they can handle different residual distributions and mean-variance relationships \citep{OHara2010, Warton2011, Szocs2015_b}.
	%
	Extending them to multi-species abundance data was thus a natural starting point for multivariate model-based analyses \citep{Warton2012}.
	%
	The univariate models are combined by summing their test statistics, which allows for inference on the whole community.
	%
	The use of MvGLM, facilitated by an easy-to-use implementation in R  \citep[in the \textit{mvabund} R-package, ][]{Wang2019}, has steadily increased within the ecological community. 
	%
	However, direct comparisons of MvGLM to other methods remain rare, with a few exceptions.
	% 
 	\citet{Warton2012} showed that MvGLMs, in contrast to algorithm-based methods, can differentiate between location (difference in mean) and dispersion (difference in mean-variance relationship) effects. 
 	%
 	%They did not consider variable selection.   
	%
	\citet{Szocs2015} found that MvGLMs performed better or similarly well in terms of statistical power than Principal Response Curves (a form of redundancy analysis) when used for the analysis of ecotoxicological semi-field studies. 
	%
	However, systematic studies of data sets with known properties are lacking and this paucity of studies hampers our capacity to make informed decisions on the selection of methods for multivariate data analysis.\\
	% CQO
% 	Another model-based method based on GLMs is the Constrained Quadratic Ordination (CQO), which is the maximum likelihood solution to Constrained Gaussian Ordination \citep[CGO,][]{Yee2004}, in which species are expected to respond unimodally to latent gradients that are linear combinations of environmental variables \citep{gauch1974ordination}.
% 	%
% 	CQO uses an extension to GLM, the \textit{Vector Generalized Linear Model} \citep[VGLM,][]{yee2003reduced}. 
% 	% CCA
% 	Before growing computation power allowed the exact estimation of CGO, Canonical Correspondence Analysis (CCA) was used as an algorithmic, heuristic approximation. 
% 	%
%     \citet{TerBraak1986} showed that the maximum likelihood estimation of CGO for Poisson-distributed counts can be approximated by correspondence analysis, given that a set of restrictive assumptions hold. 
% 	%
% 	CCA is one of the most widely used and cited multivariate statistical techniques in ecology \citep{Braak2014}.
% 	% db-RDA
% 	Another frequently used technique is Redundancy Analysis, which extends multiple linear regression to multiple response variables. 
% 	%
% 	As it assumes a linear relationship between response and environmental variable it is only appropriate for short gradients.  
% 	%
% 	Distance-based RDA (dbRDA) represents a flexible extension to RDA \citep{Legendre1999, AndersonMati2003}. 
%     %
% 	It calculates an ordination on the distance matrix of the sites-by-species  matrix $\mathbf{Y}$ constrained by the environmental variables $\mathbf{X}$. 
% 	%
% 	dbRDA was highlighted by \citet{Szocs2015}, because the possibility to use asymmetrical distance metrics and avoid the \textit{double-zero problem } \citep{Legendre2012} makes them attractive for sparse data sets. 
% 	%
% 	Generally, comparisons between these different methods are scarce and we are not aware of a study that has compared these different methods to MvGLM. 

	%------------------------------%
	%&		5. Absatz:
	%&		Aim of the Study
	%------------------------------%
	
	We compared the performance of MvGLMs to differentiate between causal and noise variables to three methods of data analysis: 
	%
	Constrained Quadratic Ordination (CQO), which is also model-based, Canonical Correspondence Analysis (CCA) and distance-based Redundancy Analysis (dbRDA), which are algorithm-based.
	%
	We applied the methods to 180 combinations of abundance data sets and explanatory variables.
	%
	The abundance data differed in distributions and sample sizes. 
	%
    Based on the assessment of a variables statistical significance, False Positive Rate (FPR) and False Negative Rates (FNR) were calculated. 



	


